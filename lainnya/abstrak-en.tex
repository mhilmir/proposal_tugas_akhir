\chapter*{ABSTRACT}
\begin{center}
  \large
  \textbf{\engtatitle{}}
\end{center}
% Menyembunyikan nomor halaman
\thispagestyle{empty}

\begin{flushleft}
  \setlength{\tabcolsep}{0pt}
  \bfseries
  \begin{tabular}{lc@{\hspace{6pt}}l}
  Student Name / NRP&: &\name{} / \nrp{}\\
  Department&: &\engdepartment{} ELECTICS - ITS\\
  Advisor&: &1. \advisor{}\\
  & & 2. \coadvisor{}\\
  \end{tabular}
  \vspace{4ex}
\end{flushleft}
\textbf{Abstract}

% Isi Abstrak
The development of drone technology and deep learning has opened new opportunities in traffic monitoring and management. Drone technology allows for the capture of images and videos from heights, providing a broad and detailed view of traffic situations. On the other hand, deep learning, as part of artificial intelligence, has the ability to analyze and understand visual data with high accuracy. This research aims to implement deep learning techniques in counting vehicle queues in traffic using drones. By using video recordings taken from drones, the deep learning model is trained to detect and count vehicles in queues. The methodology used includes data collection, data labeling, model training, and model performance evaluation. Subsequently, an estimation process is carried out to approximate the number of vehicles in the queue. The implementation of this system is expected to contribute significantly to reducing congestion and improving transportation efficiency.

\vspace{2ex}
\noindent
\textbf{Keywords: \emph{Deep Learning, Drone, Traffic, Vehicles}}