\chapter*{ABSTRAK}
\begin{center}
  \large
  \textbf{\tatitle{}}
\end{center}
\addcontentsline{toc}{chapter}{ABSTRAK}
% Menyembunyikan nomor halaman
\thispagestyle{empty}

\begin{flushleft}
  \setlength{\tabcolsep}{0pt}
  \bfseries
  \begin{tabular}{ll@{\hspace{6pt}}l}
  Nama Mahasiswa / NRP&:& \name{} / \nrp{}\\
  Departemen&:& \department FTEIC - ITS\\
  Dosen Pembimbing&:& 1. \advisor{}\\
  & & 2. \coadvisor{}\\
  \end{tabular}
  \vspace{4ex}
\end{flushleft}
\textbf{Abstrak}

% Isi Abstrak
Perkembangan teknologi drone dan deep learning telah membuka peluang baru dalam pemantauan dan pengelolaan lalu lintas. Teknologi drone memungkinkan pengambilan gambar dan video dari ketinggian yang memberikan pandangan luas dan mendetail tentang situasi lalu lintas. Di sisi lain, deep learning, sebagai bagian dari kecerdasan buatan, memiliki kemampuan untuk menganalisis dan memahami data visual dengan akurasi tinggi. Penelitian ini bertujuan untuk mengimplementasikan teknik deep learning dalam menghitung antrian kendaraan pada lalu lintas menggunakan drone. Dengan menggunakan rekaman video yang diambil dari drone, model deep learning dilatih untuk mendeteksi dan menghitung kendaraan yang berada dalam antrian. Metodologi yang digunakan mencakup pengumpulan data, pelabelan data, pelatihan model, serta evaluasi performa model. Selanjutnya dilakukan proses estimasi untuk memperkirakan jumlah antrian kendaraan yang ada. Implementasi sistem ini diharapkan dapat memberikan kontribusi signifikan dalam mengurangi kemacetan dan meningkatkan efisiensi transportasi.

\vspace{2ex}
\noindent
\textbf{Kata Kunci: \emph{Deep Learning, Drone, Kendaraan, Lalu Lintas}}