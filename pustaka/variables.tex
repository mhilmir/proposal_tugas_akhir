% Atur variabel berikut sesuai namanya

% nama
\newcommand{\name}{Mochammad Hilmi Rusydiansyah}
% \newcommand{\authorname}{Musk, Elon Reeve}
\newcommand{\authorname}{Mochammad Hilmi Rusydiansyah}
\newcommand{\nickname}{Hilmi}
\newcommand{\advisor}{Muhtadin, S.T., M.T.}
\newcommand{\coadvisor}{Dr. Rudy Dikairono, S.T., M.T.}
\newcommand{\examinerone}{-}
\newcommand{\examinertwo}{-}
\newcommand{\examinerthree}{-}
\newcommand{\headofdepartment}{Dr. Supeno Mardi Susiki Nugroho, S.T., M.T.}

% identitas
\newcommand{\nrp}{5024211008}
\newcommand{\advisornip}{19810609200912 1 003}
\newcommand{\coadvisornip}{198103252005011002}
\newcommand{\examineronenip}{-}
\newcommand{\examinertwonip}{-}
\newcommand{\examinerthreenip}{-}
\newcommand{\headofdepartmentnip}{19700313199512 1 001}

% judul
\newcommand{\tatitle}{Implementasi \emph{Deep Learning} Dalam Perhitungan Antrian Kendaraan Pada Lalu Lintas Menggunakan Drone}
\newcommand{\engtatitle}{\emph{Implementation of Deep Learning in Calculating Vehicle Queues in Traffic Using Drones}}

% tempat
\newcommand{\place}{Surabaya}

% jurusan
\newcommand{\studyprogram}{Teknik Komputer}
\newcommand{\engstudyprogram}{Computer Engineering}

% fakultas
\newcommand{\faculty}{Fakultas Teknologi Elektro dan Informatika Cerdas}
\newcommand{\engfaculty}{Faculty of Intelligent Electrical and Informatics Technology}

% singkatan fakultas
\newcommand{\facultyshort}{FTEIC}
\newcommand{\engfacultyshort}{ELECTICS}

% departemen
\newcommand{\department}{Departemen Teknik Komputer}
\newcommand{\engdepartment}{Computer Engineering Department}

% kode mata kuliah
\newcommand{\coursecode}{EC234701}
